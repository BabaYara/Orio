\begin{abstract}
% between 75 and 150 words
%The growing demand for higher levels of detail and accuracy in results means
%that the size and complexity of scientific computations is increasing at
%least as fast as the improvements in processor technology. 
%Programming
%scientific applications is hard, and optimizing them for high performance is
%even harder.  
The development of optimized codes is time-consuming and requires extensive
architecture, compiler, and language expertise, therefore, computational
scientists are often forced to choose between investing considerable time in
tuning code or accepting lower performance.
%
In this paper, we describe the first steps toward a fully automated system
for the optimization of the matrix algebra kernels that are a foundational
part of many scientific applications.  To generate highly optimized code from
a high-level MATLAB prototype, we define a three-step approach.  To begin, we
have developed a compiler that converts a MATLAB script into simple C code.
We then use the polyhedral optimization system PLuTo to optimize that code
for coarse-grained parallelism and locality simultaneously. Finally, we
annotate the resulting code with performance-tuning directives and use the
empirical performance-tuning system Orio to generate many tuned versions of
the same operation using different optimization techniques, such as loop
unrolling and memory alignment. Orio performs an automated empirical search
to select the best among the multiple optimized code variants. We discuss
performance results on two architectures.
%showing that the code generated by using our system
%significantly outperforms not only the original simple C code but also code
%based on source BLAS, ATLAS-optimized BLAS, and Intel MKL routines.
\keywords{MATLAB, code generation, empirical performance tuning}
\end{abstract}

