\section{Related Work}
\label{sec:related}

%Optimization abstraction:
%- POET (Qing Yi)
%- Apan Qasem's thesis work (Ken Kennedy's student)
%- ...
%
%Automated performance tuning:
%- ATLAS
%- SPIRAL
%- FFTW
%-...

Ideally, a developer should only have to specify a few simple command-line
options and then rely on the compiler to optimize the performance of an
application on any architecture. Compilers alone, however, cannot fully
satisfy the performance needs of scientific applications.  First,
compilers must operate in a black-box fashion and at a very low level,
limiting both the type and number of optimizations that can be done.  Second,
static analysis of general-purpose languages, such as C, C++, and Fortran, is
necessarily conservative, thereby precluding many possible optimizations.
%Third, in the process of transforming a mathematical model into a computer
%program, much potentially useful (for optimization purposes) information is
%lost since it cannot be represented by the programming language.  
Finally, extensive manual tuning of a code may prevent certain compiler optimizations
and result in worse performance on new architectures, resulting in 
loss of performance portability.

%As briefly discussed in Section~\ref{sec:motivation}, performance tuning is
%generally approached in three ways: by performing manual optimizations of key
%portions of the code; by using compiler-based source transformation tools for
%loop optimizations; and by using tuned libraries for key numerical
%algorithms.
%% Libraries
An alternative to manual or automated tuning of application codes is the use
of tuned libraries. The two basic approaches to supplying high-performance
libraries include providing a library of hand-coded options (e.g.,
\cite{BLAS,ESSL,Goto:2006fk}) and generating optimized code automatically
for the given problem and machine parameters. 
ATLAS~\cite{atlas_sc98}
%,WN147} 
for BLAS and some LAPACK
%~\cite{BLAS} and LAPACK~\cite{laug}
routines, OSKI~\cite{OSKI} for sparse linear algebra,
PHiPAC~\cite{bilmes97optimizing} for matrix-matrix products, and
domain-specific libraries such as FFTW~\cite{frigo98} and
SPIRAL~\cite{SPIRAL} are all examples of the latter approach. Most 
automatic tuning approaches perform empirical
parameter searches on the target platform.  
%FFTW uses a combination of static
%models and empirical techniques to optimize FFTs. SPIRAL generates optimized
%digital signal processing libraries by an extensive empirical search over
%implementation variants.  GotoBLAS~\cite{Goto:2006fk,Goto:fk}, on the other
%hand, achieves great performance results on several architectures by using
%hand-tuned data structures and kernel operations.  
These auto- or hand-tuned
approaches can deliver performance that can be five times as fast as that
produced by many optimizing compilers \cite{WN147}.  The library approach,
however, is limited by the fact that optimizations are highly problem- and
machine-dependent. Furthermore, at this time, the functionality of the
currently available automated tuning systems is quite limited.

%% Other annotation-based source transformation approaches
General-purpose tools for optimizing loop performance are also available.
LoopTool~\cite{LoopTool} supports annotation-based loop fusion,
unroll/jamming, skewing and tiling.  The Matrix Template Library
\cite{Siek:1998ys} uses template metaprograms to tile at both the register
and cache levels.  A new tool, POET~\cite{POET} also supports a number of
loop transformations. 
%POET offers a complex template-based syntax for
%defining transformations in a language-independent manner. 
Other research efforts whose goal, at least in part, is to enable
optimizations of source code to be augmented with performance-related
information include the X language~\cite{XLanguage} (a macro C-like language
for annotating C code), the Broadway~\cite{broadway} compiler, and
telescoping languages~\cite{telescopingurl}.
%,teleoverview,Ken99}
%, and various
%meta-programming techniques~\cite{veldhuizen95,weise93,kiczales91,chiba95}.

\comment{
Emerging annotation-based tools are normally designed by compiler researchers
and thus the interfaces are not necessarily based on concepts accessible to
computational scientists. The complexity of existing annotation languages and lack
of common syntaxes for transformations (e.g., loop unrolling) result
in steep learning curves and the inability to take advantage of more than one
approach at a time. Furthermore, at present, there is no good way for
\emph{users} to learn about the tools available and compare their
capabilities and performance.
}
