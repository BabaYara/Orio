In many scientific applications, significant time is spent in tuning codes for a particular high-performance architecture. Tuning approaches range from the relatively nonintrusive (e.g., by using compiler options) to extensive code modifications that attempt to exploit specific architecture features. Intrusive techniques often result in code changes that are not easily reversible, which can negatively impact readability, maintainability, and performance on different architectures. We introduce an extensible annotation-based empirical tuning system called Orio, which is aimed at improving both performance and productivity by enabling software developers to insert annotations in the form of structured comments into their source code that trigger a number of low-level performance optimizations on a specified code fragment.  To maximize the performance tuning opportunities, the annotation processing infrastructure is designed to support both architecture-independent and architecture-specific code optimizations. Given the annotated code as input, Orio generates many tuned versions of the same operation and empirically evaluates the alternatives to select the best performing version for production use. We have also enabled the use of the PLuTo automatic parallelization tool in conjunction with Orio to generate efficient OpenMP-based parallel code. We describe our experimental results involving a number of computational kernels, including dense array and sparse matrix operations.
