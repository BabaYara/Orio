\section{Conclusions and Future Work}
\label{sec:conclusions}

%Future directions:
%- Support for Fortran
%- Supports for various architecture-specific optimizations (e.g., SSE)
%- Automatic tuning for various problem sizes
%- Automatic parallelization for multi-core machines
%- Tuning MPI code (communication minimization)
%- Tuning for GPGPU

%- Build a decision tree that automatically directs to the best tuned code for a given problem size

We have described the design and implementation of Orio, an extensible Python
software system for defining annotation-based performance-improving
transformations. Our experiments with a number of different types of
computations on two different architectures show that Orio can deliver
performance improvements when used alone or in conjunction with other source
transformation tools.

Orio is a new tool under active development; future work includes (but is not
limited to) providing support for annotating and generating Fortran code,
defining new annotation languages and corresponding transformation modules,
e.g., using matrix notation for linear algebra operations, and integration
with other source transformation tools through new optimization modules.

\vspace{-.1in} 
\paragraph{Acknowledgments.} We would like to thank
Uday Bondhugula of Ohio State University for many productive discussions and
his valuable help with PLuTo.

\vspace{-.1in} 
